\documentclass{icsartcn}
\makeindex

\usepackage{listings}
\usepackage{amsmath}
\usepackage{dingbat}
\usepackage[linesnumbered,ruled,vlined]{algorithm2e}

\usepackage{eso-pic}
\newcommand\BackgroundPic{%
\put(0,0){%
\parbox[b][\paperheight]{\paperwidth}{%
\vfill
\centering
\includegraphics[width=\paperwidth,height=\paperheight,%
keepaspectratio]{back.png}%
\vfill
}}}

\begin{document}
\AddToShipoutPicture*{\BackgroundPic}
\title{南京师范大学}                % 标题。上限 17 个汉字宽度,不可以换行。
\subtitle{高性能计算机集群系统}             % 可选副标题。上限 20 个汉字,不可以换行。
\subsubtitle{投标文件}             % 可选副标题。上限 20 个汉字,不可以换行。
                        % 正副标题一共上限 30 个汉字。
%\date{}                % 日期。
                        % 括弧内留空:封面不出现日期。
                        % 注释掉此行:日期为编译当天。

\author{}               % 责任作者姓名。限一人。多作者时填在“修订记录”中。
\docattr{
         docid={TR201710036},        % 文档编号
%         relatedid={},             % 关联文档编号,可填多个,逗号隔开。
         email={},                  % 责任作者邮箱地址。
%         classification={内部},    % 密级。
         type={研究报告},                   % 文档分类。
         status={}}                 % 文档状态。
\maketitle

\begin{abstract}         %  文档摘要
\end{abstract}

\begin{revisions} % 修订记录,格式如下:
 v1 & 2017.09.20 & 程硕 & 报告整体完成 \\
 v2 & 2017.09.30 & 曹春 & 完善4.4节系统设计 \\
 v3 & 2017.10.08& 程硕 & 完善4.5节实验评估 \\
 v4 & 2017.10.20 & 曹春 & 整体细节修正 \\
\end{revisions}



\frontmatter            % 封面页
\tableofcontents        % 目录页
%\clearpage\listoffigures\listoftables\lstlistoflistings  % 根据需要,可以有插图目录、表的目录和代码目录

%%% 自定义命令 %%%
\renewcommand\labelenumi{\theenumi)} % 重新定义enumerate格式 1)...
\newcommand\boldred[1]{\textcolor{red}{\textbf{#1}}}
\newcommand{\myparagraph}[1]{\paragraph{#1}\mbox{}\\\indent}
\setcounter{tocdepth}{4}
\setcounter{secnumdepth}{4}
%%% 自定义命令 %%%


\mainmatter
% 正文从这里开始。

\section{摘要}
很高兴又恢复了V6访问,现在有一些界面整改的想法,想征集下群众的意见。
1.怀旧:保持老的界面,优化各个界面。
2.迎新:使用最新的nexusphp架构,类似其他PT站。
3.其他意见,欢迎留言。

% ==============================================================================
% ===================== end of section =========================================
% ==============================================================================

\section{研究背景介绍}
\subsection{概念与内涵}
一直从事兵器研究的邪恶的克拉托夫教授欲用生物兵器达成自己的野心,在俄罗斯境内的各大城市制造了一系列
大规模爆炸恐怖袭击。为了消灭邪恶势力,拯救面临生物兵器毁灭的国家,俄罗斯国防部派出了冷战时期创建的
名为“爱国者”的超级英雄团队,各个英雄异于常人,身怀特技,与号称统 治世界的邪恶教授展开了一场殊死较量。


\subsection{需求分析}

一直从事兵器研究的邪恶的克拉托夫教授欲用生物兵器达成自己的野心,在俄罗斯境内的各大城市制造了一系列
大规模爆炸恐怖袭击。为了消灭邪恶势力,拯救面临生物兵器毁灭的国家,俄罗斯国防部派出了冷战时期创建的
名为“爱国者”的超级英雄团队,各个英雄异于常人,身怀特技,与号称统 治世界的邪恶教授展开了一场殊死较量。


\subsection{国内外研究概况与发展分析}

一直从事兵器研究的邪恶的克拉托夫教授欲用生物兵器达成自己的野心\cite{kandukuri2009cloud},在俄罗斯境内的各大城市制造了一系列
大规模爆炸恐怖袭击。为了消灭邪恶势力,拯救面临生物兵器毁灭的国家,俄罗斯国防部派出了冷战时期创建的
名为“爱国者”的超级英雄团队,各个英雄异于常人,身怀特技,与号称统 治世界的邪恶教授展开了一场殊死较量。

\section{总结}
一直从事兵器研究的邪恶的克拉托夫教授欲用生物兵器达成自己的野心,在俄罗斯境内的各大城市制造了一系列
大规模爆炸恐怖袭击。为了消灭邪恶势力,拯救面临生物兵器毁灭的国家,俄罗斯国防部派出了冷战时期创建的
名为“爱国者”的超级英雄团队,各个英雄异于常人,身怀特技,与号称统 治世界的邪恶教授展开了一场殊死较量。


% ==============================================================================
% ===================== end of section =========================================
% ==============================================================================

% \LaTeX\cite{oetiker1995not} is great!
% \input{introduction} \input{problem} % 建议章节用独立的 tex 文件。
\appendix
\bibliographystyle{ieeetr}
\bibliography{main}
\printindex{}
\backmatter
\end{document}
