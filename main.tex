\documentclass{icsartcn}
\makeindex

\usepackage{listings}
\usepackage{amsmath}
\usepackage{dingbat}
\usepackage[linesnumbered,ruled,vlined]{algorithm2e}

\begin{document}
\title{国家重点研发计划重点专项}                % 标题。上限 17 个汉字宽度,不可以换行。
\subtitle{项目实施方案}             % 可选副标题。上限 20 个汉字,不可以换行。
                        % 正副标题一共上限 30 个汉字。
%\date{}                % 日期。
                        % 括弧内留空:封面不出现日期。
                        % 注释掉此行:日期为编译当天。

\author{}               % 责任作者姓名。限一人。多作者时填在“修订记录”中。
\docattr{
         docid={2017YFB1001800},        % 文档编号
%         relatedid={},             % 关联文档编号,可填多个,逗号隔开。
         email={},                  % 责任作者邮箱地址。
         classification={公开},    % 密级。
         type={研究报告},                   % 文档分类。
         status={}}                 % 文档状态。
\maketitle

\newpage


\large
\begin{center}
\bf\Large {项目实施方案编写说明}
\end{center}

项目实施方案是项目实施过程中的重要文档。编制项目实施方案须以项目任务书为基础,要求目标明确,针对性强,系统性表征突出,成果形态明确。各项目在编制实施方案时,可根据自身特点适当进行调整。如下要求供编制实施方案时参考使用。

1. 建立完整技术指标体系,明确核心指标。重大共性关键技术、应用示范类项目技术指标要细化到研究基本单元;基础前沿类项目需明确具体的项目科学目标,准确凝练需解决的所有关键技术问题或科学问题。

2. 确立明确、清晰的任务(课题)接口关系。围绕总目标,合理进行任务分解,体现项目整体性和一体化组织实施的要求。

3. 拟定项目详细的技术路线,制定合理的进度计划,设置关键节点。结合标志性成果,确定阶段考核的主要方式、方法。按照总体进度要求,各项目应对各课题研究进展提出明确要求。

4. 明确项目成果形态。提出包括成果形式、技术指标、技术成熟度、成果测试等在内的完整的成果状态表述,建立相应的检查或考核办法,确保项目阶段目标和总体目标的实现。
基础研究类项目可参照上述要求执行,实现项目科学目标。

5. 结合项目特点,建立有权威、执行力高、操作性强的项目实施组织管理机制。对实施过程中的政策、管理、技术和知识产权等风险进行充分的分析和预判,制定针对性的措施与办法;加强实施过程中的交流和检查,保证经费、人员的合理调度与使用。


\frontmatter            % 封面页
\normalsize
\tableofcontents        % 目录页
%\clearpage\listoffigures\listoftables\lstlistoflistings  % 根据需要,可以有插图目录、表的目录和代码目录

%%% 自定义命令 %%%
\renewcommand\labelenumi{\theenumi)} % 重新定义enumerate格式 1)...
\newcommand\boldred[1]{\textcolor{red}{\textbf{#1}}}
\newcommand{\myparagraph}[1]{\paragraph{#1}\mbox{}\\\indent}
\setcounter{tocdepth}{4}
\setcounter{secnumdepth}{4}
%%% 自定义命令 %%%


\mainmatter
% 正文从这里开始。

\section{项目概要}
云计算、大数据技术和“人-机-物”三元融合应用模式不断加速社会的信息化进程,其软件基础设施面临环境资源不断变迁的挑战。在此背景下,亟需以智能化为手段、以可成长为本质的可持续演化软件理论、方法和技术。

为此,本项目科学问题定位于:

1. \textbf{可成长软件的架构模型与运行机理:} 给出内化了自适应和持续演化能力的可成长软件的系统形态,阐明其利用感知、学习和数据适应环境资源变迁,不断演进成长的机制原理。

2. \textbf{复杂软件系统成长能力挖掘与增强:} 显式化既有软件系统的环境资源依赖和应用意图,基于软件定义集成自适应和持续演化机制,赋予其成长能力。

项目研究目标在于建立网构化可成长软件范型理论,给出相应方法技术,研制支撑平台;针对军事指挥、智能电网和无人机等关键领域场景开发解决方案,进行应用验证;形成一套面向持续演化的智能化软件理论、方法和技术。项目将重点研究:

\textbf{1. 可持续演化的智能化网构软件范型理论与方法学框架}

发展网构软件范型理论,给出可成长软件的系统形态和构造方法,研究内化自适应与持续演化能力的软件架构模型,以及基于感知、学习和程序数据资源等智能化手段的实现原理,提供可持续演化软件系统的服务质量评估方法与保障机制,形成面向持续演化的网构化软件开发方法与支撑技术体系。

\textbf{2. 面向自主适应的软件情境感知、适应决策与在线更新}

研究多模态软件情境感知与建模技术,突破环境上下文一致性检测与修复的性能瓶颈;研究软件资源依赖的建模与分析、资源使用的在线监控与优化技术,实现基于学习和在线验证的自适应决策;研究高一致、低干扰的软件在线更新技术,支持可靠高效的软件热部署与动态更新。

\textbf{3. 面向过程演进的数据驱动式软件自动构造与演化方法}

研究基于表示学习的编程语言模型,给出基于深度神经网络的程序代码生成技术,实现面向错误修正和功能增强的软件自动化;研究基于程序分析的应用资源使用分析、软件配置理解和影响域分析等技术,实现软件资源依赖与应用意图关系的自动获取,以此指导软件持续演化。

\textbf{4. 深度软件定义的自适应与持续演化软件系统支撑平台}

研究软件运行环境的深度软件定义机制和支撑长期生存软件应用的基础设施虚拟化架构,实现可编程的自适应应用容器;研究基于模型的持续演化策略构造及验证机制,支持应用意图的动态维持;研究细粒度的资源调度优化、智能化的动态重配置,实现应用意图导向的演化实施。

\textbf{5. 复杂关键应用场景下的软件持续演化应用验证与示范}

针对军事指挥控制、智能电网,以及无人机和列控系统等领域的关键应用场景,开发软件自适应与持续演化的领域解决方案,进行应用验证。

项目团队由中科院院士吕建教授领衔,成员包括国家杰出青年基金获得者3名、长江学者3名、CCF青年科学家2名等;项目团队长期合作,在面向互联网的软件方法与技术、云计算与服务计算、可信软件等方面取得了一系列的重要成果,近五年来,先后获得国家自然科学二等奖和国家科技进步和技术发明一、二等奖共7项,可为本项目提供良好的前期工作基础。

项目预期成果包括:一种可持续演化的智能化网构软件范型理论与方法学框架;两个方面的关键技术,即以准确及时的情境感知、智能可信的适应决策和高效可靠的在线更新为特征的软件自适应技术,和以环境资源依赖分析与数据驱动软件构造演化为基础的软件过程演进技术;一套深度软件定义的自适应与持续演化软件系统支撑平台;面向军事指挥、智能电网和无人机等关键领域应用场景的软件自适应和持续演化解决方案等;最终形成一套可持续演化软件理论、方法和技术体系,取得重要的国际学术影响,为我国软件产业跨越发展和社会信息化进程提供系统化的原创技术支撑。

% ==============================================================================
% ===================== end of section =========================================
% ==============================================================================


\section{项目课题分解及主要研究工作}

\section{项目实施关键节点与具体实施计划}

\section{项目组织管理机制}

\section{项目成果呈现形式及测试方法}


% ==============================================================================
% ===================== end of section =========================================
% ==============================================================================

\LaTeX\cite{liu2009greencloud} is great!
% \input{introduction} \input{problem} % 建议章节用独立的 tex 文件。
\appendix
\bibliographystyle{ieeetr}
\bibliography{main}
\printindex{}
\end{document}
